\documentclass{article}
\usepackage[utf8]{inputenc}

\title{18.304 Final Project \\  Hadamard Matrices}
\author{Nicolas Bravo \\ nbravo@mit.edu }
\date{May 2015}

\usepackage{natbib}
\usepackage{graphicx}
\usepackage{amsmath}
\usepackage{amsthm}

\newtheorem{theorem}{Theorem}[section]
\newtheorem{corollary}{Corollary}[theorem]
\newtheorem{lemma}{Lemma}[theorem]
\theoremstyle{definition}
\newtheorem{definition}{Definition}[section]

%\renewcommand\qedsymbol{$\blacksquare$}

\begin{document}

\maketitle

\section{Abstract}

\section{Introduction}

\begin{definition}
 A Hadamard matrix is a square matrix whose entries are either +1 or -1 and whose rows are mutually orthogonal.
\end{definition}

\section{Construction}
There are several ways to construct Hadamard matrices. For example, James Joseph Sylvester proposed the following: Let $H$ be a Hadamard matrix of order $n$. Then
\begin{equation*}
\begin{bmatrix}
H & H \\
H & -H
\end{bmatrix}
\end{equation*}
is a Hadamard matrix of order $2n$. This construction could lead to the following sequence of Hadamard matrices:
$H_1 =
\begin{bmatrix}
1
\end{bmatrix},
$
$H_2 =
\begin{bmatrix}
1 & 1 \\
1 & -1 \\
\end{bmatrix},
$
$H_{2^k} =
\begin{bmatrix}
H_{2^{k-1}} & H_{2^{k-1}} \\
H_{2^{k-1}} & H_{2^{k-1}}
\end{bmatrix}
$

\section{Equality of Hadamard Matrices}

\section{Results}
\begin{theorem}
Let $H$ be a Hadamard matrix of order $n$. Then $HH^T = nI_n$.
\end{theorem}

\begin{proof}
Since each entry in $H$ is $\pm 1$, we know that the length of each row vector is $\sqrt{n}$. Further, we know from the definition of Hadamard matricecs that each row is orthogonal to each other row, so if we divide $H$ by $\sqrt n$ we obtain an orthognal matrix $Q = \frac{1}{\sqrt n} H$. We then see that
\begin{align*}
  QQ^T &= I_n\\
  \left(\frac{1}{\sqrt n} H\right)\left(\frac{1}{\sqrt n} H^T\right) &= I_n\\
  HH^T &= nI_n
\end{align*}
\end{proof}

\begin{theorem}
If $H$ is an $n \times n$ Hadamard matrix, then $n = 1$ or $n = 2$ or $n \equiv 0 \mod 4$.
\end{theorem}

\begin{proof}
\end{proof}

\begin{theorem}
There exists an $n \times n$ matrix with entries $\pm 1$ whose determinant is greater than $\sqrt{n!}$
\end{theorem}

\begin{proof}
\end{proof}
\section{Applications}

\section{Current Research}

\section{Conclusion}

\end{document}

